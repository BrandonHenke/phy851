\documentclass[a4paper,12pt,twoside]{article}

\usepackage{preamble}

\title{Homework 05}
\author{Brandon Henke\\PHY851\\Scott Pratt}
\date{September 20, 2021}

\setcounter{section}{3}

\fancyhead[LE,RO]{B. Henke}
\fancyhead[RE,LO]{\thepage}

\begin{document}
%\tableofcontents

\maketitle
% \begin{multicols*}{2}

\setcounter{subsection}{4}
\subsection{}%3.5
\begin{align}
	\vb{B} &= B\vu{z},\\
	\vb{E} &= 0, \qq{and}\\
	\vb{A} &= Bx\vu{y}.
\end{align}
For $\vb{E} = 0$ one can assume $A_0 = \Phi = 0$.
Boosting at $v_y$ in the $\vu{y}$ direction gives
\begin{align}
	A_0' &= \gamma\left(A_0 - \bm{\beta}\vdot\vb{A}\right),\\
	&= - \gamma v_y B x.
\end{align}
For nonrelativistic velocities, $\gamma \approx 1$, so,
\begin{equation}
	A_0' = -v_y Bx.
\end{equation}
Thus
\begin{equation}
	\vb{E}' = -\grad{A_0'} = v_y B \vu{x}.
\end{equation}

\setcounter{section}{4}
\setcounter{subsection}{0}
\subsection{}%4.1
The angular momentum operator for the $\vu{z}$ axis is given by
\begin{equation}
	\hat{L}_3 = \hat{r}_1\hat{p}_2-\hat{r}_2 \hat{p}_1.
\end{equation}
From this, the commutator with $\hat{r}^2$ is
\begin{align}
	\comm{\hat{r}^2}{\hat{L}_3}
	&= \hat{r}_1\hat{r}^2\hat{p}_2 - \hat{r}_2\hat{r}^2\hat{p}_1 - \hat{r}_1\hat{p}_2\hat{r}^2 + \hat{r}_2\hat{p}_1\hat{r}^2,\\
	&= \hat{r}_1\comm{\hat{r}^2}{\hat{p}_2}-\hat{r}_2\comm{\hat{r}^2}{\hat{p}_1},\\
	&= i\hbar \hat{r}_1\hat{r}_2 - i\hbar\hat{r}_1\hat{r}_2,\\
	&= 0.
\end{align}
Hence, the two operators commute.

\subsection{}%4.2
Let $i = \sigma_{123} = \sigma_1\sigma_2\sigma_3$, and
\begin{align}
	\alpha &= \sum_{n=1}^3 \alpha_n\sigma_n,\\
	\beta &= \sum_{n=1}^3 \beta_n\sigma_n, \qq{and}\\
	\gamma &= \sum_{n=1}^3 \gamma_n\sigma_n.
\end{align}
Then
\begin{align}
	e^{i\alpha/2}e^{i\beta/2} &= \left(\cos(\frac{\abs{\alpha}}{2}) + i\sin(\frac{\abs{\alpha}}{2})\frac{\alpha}{\abs{\alpha}}\right)
	\left(\cos(\frac{\abs{\beta}}{2}) + i\sin(\frac{\abs{\beta}}{2})\frac{\beta}{\abs{\beta}}\right),\\
	&= \cos(\frac{\abs{\alpha}}{2})\cos(\frac{\abs{\beta}}{2})
	- \frac{\alpha\beta}{\abs{\alpha}\abs{\beta}} \sin(\frac{\alpha}{2})\sin(\frac{\abs{\beta}}{2})\\
	&+ i\left(\frac{\alpha}{\abs{\alpha}} \sin(\frac{\abs{\alpha}}{2}) \cos(\frac{\abs{\beta}}{2})
	+ \frac{\beta}{\abs{\beta}} \sin(\frac{\abs{\alpha}}{2}) \cos(\frac{\abs{\beta}}{2})\right).
\end{align}
By definition of the geometric product, $\alpha\beta = \alpha\cdot\beta + i(\alpha\cross\beta)$.
Thus
\begin{align}
	e^{i\alpha/2}e^{i\beta/2}
	&= \cos(\frac{\abs{\alpha}}{2})\cos(\frac{\abs{\beta}}{2})-\frac{\alpha\cdot\beta}{\abs{\alpha}\abs{\beta}} \sin(\frac{\alpha}{2})\sin(\frac{\abs{\beta}}{2})\nonumber\\
	&+ i\left[ \frac{\alpha}{\abs{\alpha}} \sin(\frac{\abs{\alpha}}{2}) \cos(\frac{\abs{\beta}}{2})
	+ \frac{\beta}{\abs{\beta}} \sin(\frac{\abs{\alpha}}{2}) \cos(\frac{\abs{\beta}}{2}) \right.\nonumber\\
	&+ \left. \frac{\alpha \cross \beta}{\abs{\alpha}\abs{\beta}} \sin(\frac{\alpha}{2})\sin(\frac{\abs{\beta}}{2})\right]
\end{align}
If one expands $e^{i\gamma/2}$ in the same way and equates the scalar and bivector parts, then one gets the desired relationships:
\begin{align}
	\cos(\frac{\abs{\gamma}}{2}) &= \cos(\frac{\abs{\alpha}}{2})\cos(\frac{\abs{\beta}}{2})
	- \sin(\frac{\abs{\alpha}}{2})\sin(\frac{\abs{\beta}}{2}).\\
	\frac{\gamma}{\abs{\gamma}} \sin(\frac{\abs{\gamma}}{2}) &= \frac{\alpha}{\abs{\alpha}} \sin(\frac{\abs{\alpha}}{2}) \cos(\frac{\abs{\beta}}{2})
	+ \frac{\beta}{\abs{\beta}} \sin(\frac{\abs{\alpha}}{2}) \cos(\frac{\abs{\beta}}{2})\nonumber\\
	&+ \frac{\alpha \cross \beta}{\abs{\alpha}\abs{\beta}} \sin(\frac{\alpha}{2})\sin(\frac{\abs{\beta}}{2})
\end{align}
\subsection{}%4.3
\subsubsection{}
\begin{align}
	\left[S_3,S_1\right] &= S_3S_1 - S_1S_3,\\
	&= \frac{\hbar^2}{\sqrt{2}}\left[
		\begin{pmatrix}
			0 & 1 & 0 \\
			0 & 0 & 0 \\
			0 & -1 & 0
		\end{pmatrix}
		+ \begin{pmatrix}
			0 & 0 & 0 \\
			-1 & 0 & 1 \\
			0 & 0 & 0
	\end{pmatrix}
	\right],\\
	&= i\hbar\epsilon_{312}S_2.
\end{align}
\subsubsection{}
\begin{align}
	\sum_{n=1}^{3} S_n^{2} &= \hbar^2 \left[
		\frac{1}{2}\begin{pmatrix}
			1 & 0 & 1\\
			0 & 2 & 0\\
			1 & 0 & 1
		\end{pmatrix}
		+ \frac{1}{2}\begin{pmatrix}
			1 & 0 & -1\\
			0 & 2 & 0 \\
			-1 & 0 & 1
		\end{pmatrix}
		+ \begin{pmatrix}
			1 & 0 & 0\\
			0 & 0 & 0\\
			0 & 0 & 1
		\end{pmatrix}
	\right],\\
	&= 2\hbar^2 \hat{1}.
\end{align}
\subsection{}%4.4
This is just a rotation of the $\hat{x}$ operator about the $\vu{z}$ axis:
\begin{align}
	\hat{x}(\phi) &= e^{i\hat{L}_3\phi/2}\hat{x}e^{-i\hat{L}_3\phi/2},\\
	&= \hat{x}\cos\phi+\hat{y}\sin\phi.
\end{align}
\subsection{}%4.5
\begin{equation}
	A = \begin{pmatrix}
		 1 & 2 & 3 & 4 & 5 & 6 \\
		 2 & 3 & 1 & 5 & 6 & 4 \\
		 3 & 1 & 2 & 6 & 4 & 5 \\
		 4 & 6 & 5 & 1 & 3 & 2 \\
		 5 & 4 & 6 & 2 & 1 & 3 \\
		 6 & 5 & 4 & 3 & 2 & 1 \\
	\end{pmatrix}
\end{equation}




% \nocite{*}
% \printbibliography[title={References},heading=bibnumbered]


% \end{multicols*}
\end{document}
