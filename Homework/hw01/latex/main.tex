\documentclass[a4paper,12pt,twoside]{article}

\usepackage{preamble}

\title{Homework 01}
\author{Brandon Henke\\PHY851\\Scott Pratt}
\date{September 9, 2021}

\setcounter{section}{1}

\fancyhead[LE,RO]{B. Henke}
\fancyhead[RE,LO]{\thepage}

\begin{document}
%\tableofcontents

\maketitle
\begin{multicols*}{2}


\subsection{}%1
\subsubsection*{a}
\begin{align}
    R(\phi) &= \begin{pmatrix}
        cos(\phi) & -\sin(\phi)\\
        \sin(\phi) & \cos(\phi)
    \end{pmatrix},
    \label{eq: rotationMat}\\
    R\left(\frac{\pi}{4}\right) &= \frac{1}{\sqrt{2}}\begin{pmatrix}
        1 & 1 \\ -1 & 1
    \end{pmatrix}.
\end{align}
\subsubsection*{b}
See equation (\ref{eq: rotationMat}).
\subsubsection*{c}
Right circularly polarized light is given by
\begin{equation}
    \ket{R} = \frac{1}{\sqrt{2}}\left( \ket{x}+i\ket{y} \right).
\end{equation}
\begin{equation}
    \phi(z,t) = \frac{1}{\sqrt{2}}e^{-i\omega t}\begin{pmatrix}
    e^{i k_x z} \\ e^{i k_y z}
    \end{pmatrix}.
\end{equation}
Thus,
\begin{equation}
    \ket{\psi} = \frac{1}{\sqrt{2}} \left( \ket{x} - i e^{i(k_x-k_y)Z}\ket{y} \right).
\end{equation}
\subsubsection*{d}
\begin{align}
    \rho_{\ket{R}} &= \op{R},\\
    &= \frac{1}{2} \left( \ket{x} + i\ket{y} \right)\left( \bra{x} - i\bra{y} \right),\\
    &= \frac{1}{2} \left( \mathbb{1} + \sigma_2 \right).
\end{align}
\subsubsection*{e}
\begin{equation}
    \rho_\psi = \frac{1}{2} \op{x} + \frac{1}{2} \op{y}.
\end{equation}
\subsection{}%2
\begin{align}
    \ev{R(\theta)}{x} &= \ev{(\cos\theta-i\sin\theta)}{x},\\
    &= \cos\theta - i\sin\theta\ev{\sigma_2}{x},\\
    &= \cos\theta.
\end{align}
\begin{align}
    \ev{R\left(\frac{\pi}{2}\right)}{x} &= 0,\\
    \ev{R\left(\pi\right)}{x} &= -1,\\
    \ev{R\left(2\pi\right)}{x} &= 1.
\end{align}
\subsection{}%3
\begin{align}
    &\ev{R_y(\theta)}{z,+}\\ &= \ev{\left(\cos\frac{\theta}{2}+i\sin\frac{\theta}{2}\right)}{z,+},\\
    &= \cos\frac{\theta}{2} + i\sin\frac{\theta}{2}\ev{\sigma_2}{z,+},\\
    &= \cos\frac{\theta}{2}.
\end{align}
\begin{align}
    \ev{R_y\left(\frac{\pi}{2}\right)}{z,+} &= \frac{1}{\sqrt{2}},\\
    \ev{R_y\left(\pi\right)}{z,+} &= 0,\\
    \ev{R_y\left(2\pi\right)}{z,+} &= -1.
\end{align}
\subsection{}%4
Let $U$ bet a unitary transformation.
Then
\[
    UU^\dagger = U^\dagger U = \mathbb{1}.
\]
Hence,
\[
    U^\dagger \mathbb{1} U = U^\dagger U \mathbb{1} = \mathbb{11} = \mathbb{1}.
\]
\subsection{}%5
\begin{equation}
    U^\dagger = \frac{1}{\sqrt{2}} (\mathbb{1}+i\sigma_3).
\end{equation}
\subsubsection*{a}
\begin{align}
    U\sigma_1U^\dagger &= \frac{1}{2}(\mathbb{1}-i\sigma_3)\sigma_1(\mathbb{1}+i\sigma_3),\\
    &= \frac{1}{2}(\sigma_1 - i\sigma_3\sigma_1)(\mathbb{1}+i\sigma_3),\\
    &= \frac{1}{2}(\sigma_1+i[\sigma_1,\sigma_3] - \sigma_1),\\
    &= \frac{1}{2} i (-2i\sigma_2),\\
    &= \sigma_2.
\end{align}
\subsubsection*{b}
\begin{align}
    U\ket{x,+} &= \frac{1}{2}(\mathbb{1}+i\sigma_3)(\ket{\uparrow}+\ket{\downarrow}),\\
    &= \frac{1}{2}(\ket{\uparrow}+\ket{\downarrow}-i\ket{\uparrow}+i\ket{\downarrow}),\\
    &= \frac{1}{\sqrt{2}}(\ket{\uparrow}+i\ket{\downarrow}).
    \label{eq: y eigenvec}
\end{align}
This expression (\ref{eq: y eigenvec}) is an eigenstate of $\sigma_2$.
\subsection{}%6
\begin{equation}
    U^\dagger = v_i^{(j)}.
\end{equation}
\subsubsection*{a}
\begin{align}
    UU^\dagger &= v_j^{*(i)}v_i^{(j)},\\
    &= \left\{\begin{array}{lr}
        1, & i=j \\
        0, & i\neq j
    \end{array}\right.,\\
    &= \mathbb{1}.
\end{align}
\subsubsection*{b}
\begin{align}
    U_{ij} v_i^{(n)} &= v_j^{*(i)}v_i^{(n)},\\
    &= \delta_{nj}\qq{(ortho.)}\\
    K' = UKU^\dagger:&\\
    K'U v^{(n)} &= UKU^\dagger U v^{(n)},\\
    &= UK v^{(n)},\\
    &= \lambda_n \left(Uv^{(n)}\right).
\end{align}
\subsection{}%7
\begin{equation}
    M = \begin{pmatrix}
     1 & 0 & 0 \\
     0 & 0 & 1 \\
     0 & 1 & 0 \\
    \end{pmatrix}.
\end{equation}
\subsubsection*{a}
\begin{align}
    \left| \begin{matrix}
     \lambda-1 & 0 & 0 \\
     0 & \lambda & -1 \\
     0 & -1 & \lambda
    \end{matrix}\right|\!\!\!\!\!\!\!\!\!\!\!\!&\nonumber\\
    &= (\lambda-1)(\lambda^2-1) = 0,\\
    \lambda &= 1,\pm i.
\end{align}
\subsubsection*{b}
\begin{align}
    \begin{pmatrix}
     \lambda-1 & 0 & 0 \\
     0 & \lambda & -1 \\
     0 & -1 & \lambda
    \end{pmatrix}&
    \begin{pmatrix}
        a \\ b \\ c
    \end{pmatrix}
    = \begin{pmatrix}
        0 \\ 0 \\ 0
    \end{pmatrix}.\\
    a &= 0,\\
    c &= b\lambda.\\
    v_1 &= \frac{1}{\sqrt{2}}\begin{pmatrix}
        0 \\ 1 \\ 1
    \end{pmatrix},\\
    v_i &= \frac{1}{\sqrt{2}}\begin{pmatrix}
        0 \\ 1 \\ i
    \end{pmatrix},\\
    v_{-i} &= \frac{1}{\sqrt{2}}\begin{pmatrix}
        0 \\ 1 \\ -i
    \end{pmatrix}.
\end{align}
\subsection{}%8
\begin{equation}
    K = \begin{pmatrix}
        A & C^* \\
        C & B
    \end{pmatrix}.
\end{equation}
\subsubsection*{a}
\begin{align}
    & (\lambda - A)(\lambda-B)-C^*C = 0,\\
    & \lambda_\pm = \frac{A+B \pm \sqrt{(A^2-B^2)+4C^*C}}{2}.
\end{align}
\subsubsection*{b}
\begin{align}
    \begin{pmatrix}
        \lambda-A & -C^* \\
        -C & \lambda-B
    \end{pmatrix}
    & \begin{pmatrix}
        a \\ b
    \end{pmatrix}
    = \begin{pmatrix}
        0 \\ 0
    \end{pmatrix},\\
    a(\lambda-A) - bC^* &= 0,\\
    -aC+b(\lambda-B) &=0,\\
    a &= b \sqrt{\frac{C^*(\lambda-B)}{C(\lambda-A}},
\end{align}
\begin{equation}
    v_\pm = \frac{1}{\sqrt{\frac{C^*(\lambda-B)}{C(\lambda-A}-1}}\begin{pmatrix}
            \sqrt{\frac{C^*(\lambda-B)}{C(\lambda-A}},\\
            1
        \end{pmatrix}
\end{equation}
\subsection{}%9
\begin{equation}
    \ket{\psi} = \frac{1}{2} (\sqrt{3} \ket{x} + \ket{y}).
\end{equation}
\begin{align}
    \ev{(E-\ev{E})^2}^{1/2} &= \sqrt{Np(1-p)},\\
    p &= \abs{\ip{x}{y}}^2 = \frac{3}{4},\\
    \ev{(E-\ev{E})^2}^{1/2} &= \frac{\sqrt{3N}}{4}.\\
    \ev{E} &= Np,\\
    &= \frac{3N}{4}.\\
    N &= \frac{10\mathrm{J}}{h\nu} \approx 3.32 \cdot 10^{19},\\
    \frac{\ev{(E-\ev{E})^2}^{1/2}}{\ev{E}} &= \frac{1}{\sqrt{3N}},\\
    &\approx 1.00 \cdot 10^{-10}\mathrm{J^{-1/2}}.
\end{align}

\nocite{*}
\printbibliography[title={References},heading=bibnumbered]


\end{multicols*}
\end{document}
