\documentclass[
a4paper,
10pt,
twoside,
% prd,
% aps,
% nofootinbib,
% superscriptaddress,
% floatfix,
% preprintnumbers,
]{article}

% MUST BE RUN WITH LUALATEX

\usepackage{preamble}
\usepackage{titleinfo}

\geometry{ % Set document margins
	top			= 2cm,
	bottom	= 2cm,
	left		= 1cm,
	right		= 1cm
}

\newcommand{\mcols}{2} % Choose number of columns (>= 1)


\bibSetup{refs.bib} % Give references file 
% ===== Format headers & footers =====

\pagestyle{fancy}
\fancyhf{}
\fancyhead[LE,RO]{B. Henke}
\fancyhead[LO]{\headertitle\hspace{0.5cm}}
\fancyhead[RE]{\hspace{0.5cm}\headertitle}
\fancyfoot[RE,LO]{\thepage}

\begin{document}
% \tableofcontents
\titleinf
\maketitle
\startmcols

\section{Pauli Matrices}
\subsection{}% A
\begin{align}
	\mel{+}{S_x}{+} &= \frac{\hbar}{2} \left(\ip{+}{+}\ip{-}{+} + \ip{+}{-}\ip{+}{+} \right),\\
	&= 0.\\
	\mel{-}{S_x}{+} &= \frac{\hbar}{2} \left(\ip{-}{+}\ip{-}{+} + \ip{-}{-}\ip{+}{+} \right),\\
	&= \frac{\hbar}{2}.\\
	\mel{+}{S_x}{-} &= \frac{\hbar}{2} \left(\ip{+}{+}\ip{-}{-} + \ip{+}{-}\ip{+}{-} \right),\\
	&= \frac{\hbar}{2}.\\
	\mel{-}{S_x}{-} &= \frac{\hbar}{2} \left(\ip{-}{+}\ip{-}{-} + \ip{-}{-}\ip{+}{-} \right),\\
	&= 0.\\
	S_x &= \frac{\hbar}{2}\begin{pmatrix}
		0&1\\1&0
	\end{pmatrix}.
\end{align}
\begin{align}
	\mel{+}{S_y}{+} &= \frac{i\hbar}{2} \left(-\ip{+}{+}\ip{-}{+} + \ip{+}{-}\ip{+}{+} \right),\\
	&= 0.\\
	\mel{-}{S_y}{+} &= \frac{i\hbar}{2} \left(-\ip{-}{+}\ip{-}{+} + \ip{-}{-}\ip{+}{+} \right),\\
	&= \frac{i\hbar}{2}.\\
	\mel{+}{S_y}{-} &= \frac{i\hbar}{2} \left(-\ip{+}{+}\ip{-}{-} + \ip{+}{-}\ip{+}{-} \right),\\
	&= -\frac{i\hbar}{2}.\\
	\mel{-}{S_y}{-} &= \frac{i\hbar}{2} \left(-\ip{-}{+}\ip{-}{-} + \ip{-}{-}\ip{+}{-} \right),\\
	&= 0.\\
	S_y &= \frac{\hbar}{2}\begin{pmatrix}
		0&-i\\i&0
	\end{pmatrix}.
\end{align}
\begin{align}
	\mel{+}{S_z}{+} &= \frac{\hbar}{2} \left(\ip{+}{+}\ip{+}{+} - \ip{+}{-}\ip{-}{+} \right),\\
	&= \frac{\hbar}{2}.\\
	\mel{-}{S_z}{+} &= \frac{\hbar}{2} \left(\ip{-}{+}\ip{+}{+} - \ip{-}{-}\ip{-}{+} \right),\\
	&= 0.\\
	\mel{+}{S_z}{-} &= \frac{\hbar}{2} \left(\ip{+}{+}\ip{+}{-} - \ip{+}{-}\ip{-}{-} \right),\\
	&= 0.\\
	\mel{-}{S_z}{-} &= \frac{\hbar}{2} \left(\ip{-}{+}\ip{+}{-} - \ip{-}{-}\ip{-}{-} \right),\\
	&= -\frac{\hbar}{2}.\\
	S_z &= \frac{\hbar}{2}\begin{pmatrix}
		1&0\\0&-1
	\end{pmatrix}.
\end{align}

\subsection{}% B
The eigenvalues for all of the following are $\pm 1$.
\begin{align}
	\ket{S_x,\pm} &= \frac{1}{\sqrt{2}} \left(\ket{-}\pm\ket{+}\right).\\
	\ket{S_y,\pm} &= \frac{1}{\sqrt{2}} \left(\ket{-}\pm i\ket{+}\right).\\
	\ket{S_z,\pm} &= \ket{\pm}.
\end{align}

\subsection{}% C
\begin{align}
	S_x &= \frac{\hbar}{2\sqrt{2}^2} \left(\ket{-}+\ket{+}\right)\left(\bra{-}+\bra{+}\right),\\
	&\quad- \frac{\hbar}{2\sqrt{2}^2} \left(\ket{-}-\ket{+}\right)\left(\bra{-}-\bra{+}\right),\\
	&= \frac{\hbar}{4}\left( 2\op{-}{+}+2\op{+}{-}\right),\\
	&= \frac{\hbar}{2}\left(\op{-}{+}+\op{+}{-}\right).
\end{align}
\begin{align}
	S_y &= \frac{\hbar}{2\sqrt{2}^2} \left(\ket{-}+i\ket{+}\right)\left(\bra{-}+i\bra{+}\right),\\
	&\quad- \frac{\hbar}{2\sqrt{2}^2} \left(\ket{-}-i\ket{+}\right)\left(\bra{-}-i\bra{+}\right),\\
	&= \frac{\hbar}{4}\left( 2i\op{-}{+}-2i\op{+}{-}\right),\\
	&= \frac{i\hbar}{2}\left(\op{-}{+}-\op{+}{-}\right).
\end{align}
\begin{align}
	S_z &= \frac{\hbar}{2} \op{+} - \frac{\hbar}{2}\op{-},\\
	&= \frac{\hbar}{2} \left(\op{+} - \op{-}\right).
\end{align}

\section{Sakurai 1.4}
\subsection{}% A
Since all Pauli matrices are traceless, $\tr{X}= 2a_0$.
Additionally, $\tr{\sigma_k X} = 0 + \sum_{i=1}^3\tr{\sigma_k\sigma_i} = 2a_k.$

\subsection{}% B
\begin{align}
	a_0 &= \frac{1}{2}\left(X_{00} + X_{11}\right),\\
	a_1 &= \frac{1}{2}\left(X_{10} + X_{01}\right),\\
	a_2 &= -\frac{i}{2}\left(X_{10} - X_{01}\right),\\
	a_3 &= \frac{1}{2}\left(X_{00} - X_{11}\right).
\end{align}

\section{Sakurai 1.10}
First, the Pauli matrices anti-commutate:
\begin{align}
	\sigma_1 \sigma_2 &= \begin{pmatrix}
		0&1\\1&0
	\end{pmatrix}\begin{pmatrix}
		0&-i\\i&0
	\end{pmatrix},
	\label{eq: anti-commute_first}\\
	&= \begin{pmatrix}
		i&0\\0&-i
	\end{pmatrix},\\
	&= -\sigma_2\sigma_1.
\end{align}
\begin{align}
	\sigma_2 \sigma_3 &= \begin{pmatrix}
		0&-i\\i&0
	\end{pmatrix}\begin{pmatrix}
		1&0\\0&-1
	\end{pmatrix},\\
	&= \begin{pmatrix}
		0&i\\i&0
	\end{pmatrix},\\
	&= -\sigma_3\sigma_2.
\end{align}
\begin{align}
	\sigma_3 \sigma_1 &= \begin{pmatrix}
		1&0\\0&-1
	\end{pmatrix}\begin{pmatrix}
		0&1\\1&0
	\end{pmatrix},\\
	&= \begin{pmatrix}
		-1&0\\0&1
	\end{pmatrix},\\
	&= -\sigma_1\sigma_3.
	\label{eq: anti-commute_last}
\end{align}
Here's a detail that will be used in a bit:
\begin{align}
	\sigma_1\sigma_2\sigma_3 &= \begin{pmatrix}
		0&1\\1&0
	\end{pmatrix}\begin{pmatrix}
		0&-i\\i&0
	\end{pmatrix}\begin{pmatrix}
		1&0\\0&-1
	\end{pmatrix},\\
	&= i.
	\label{eq: pseudoscalar}
\end{align}

Since $\sigma_i^2 = 1$, and $\sigma_i\sigma_j = -\sigma_j\sigma_i$ (see equations \ref{eq: anti-commute_first}-\ref{eq: anti-commute_last}) for $i\neq j$:
\begin{align}
	\anticommutator{\sigma_i}{\sigma_j} &= 2\delta_{ij}.\\
	\commutator{\sigma_i}{\sigma_j} &= 0.\quad(i=j)\\
\end{align}
Since $-i=\sigma_3\sigma_2\sigma_1$ (eq. \ref{eq: pseudoscalar}),
\begin{align}
	\commutator{\sigma_i}{\sigma_j} &= 2\sigma_i\sigma_j,\\
	&= 2\sigma_i\sigma_j(-i^2)\sigma_k^2,\quad(i\neq j \neq k)\\
	&= 2i \left[\sigma_i\sigma_j\sigma_k \sigma_3\sigma_2\sigma_1\right] \sigma_k,\\
	&= 2i \epsilon_{ijk} \sigma_k. 
\end{align}
Since $S_k = \frac{\hbar}{2}\sigma_k$, $\commutator{S_i}{S_j} = \frac{\hbar^2}{4}\commutator{\sigma_i}{\sigma_j}$, and the same for the anticommutator:
\begin{align}
	\therefore \commutator{S_i}{S_j} &= \hbar i \epsilon_{ijk} S_k.\\
	\anticommutator{S_i}{S_j} &= \frac{\hbar^2}{2} \delta_{ij}.
\end{align}

\section{Sakurai 1.11}
Consider the unit vector $n = n^k \sigma_k$.
This can be written as
\begin{equation}
	n = \sin\theta(\cos\phi \sigma_1 + \sin\phi \sigma_2) + \cos\theta \sigma_3.
\end{equation}
Let $s = \frac{\hbar}{2}n$.
The expectation value for this spin operator with the spin aligned with $n$ is $\hbar/2$:
\begin{align}
	1 = \sin\theta(\cos\phi \ev{\sigma_1} + \sin\phi \ev{\sigma_2}) + \cos\theta \ev{\sigma_3}
\end{align}

The spin state is given by
\begin{equation}
	\ket{\psi} = \alpha \ket{+} + \beta \ket{-},
\end{equation}
where $\abs{\alpha}^2 + \abs{\beta}^2 = 1$.
The expectation values for each of the Pauli matrices, in terms of $\alpha$ and $\beta$, are
\begin{align}
	\ev{\sigma_1} &= \alpha\beta^* + \alpha^* \beta,\\
	\ev{\sigma_2} &= i(\alpha\beta^* - \alpha^* \beta),\\
	\ev{\sigma_3} &= \alpha\alpha^* + \beta\beta^*.
\end{align}
These are equal to the respective components of the unit vector, $n$:
\begin{align}
	\sin\theta\cos\phi &= \alpha\beta^* + \alpha^* \beta,\\
	\sin\theta\sin\phi &= i(\alpha\beta^* - \alpha^* \beta),\\
	\cos\theta &= \alpha\alpha^* + \beta\beta^*.
\end{align}
Solving this for $\alpha$ and $\beta$ gives (up to a global phase)
\begin{align}
	\alpha &= \cos\frac{\theta}{2},\\
	\beta &= \sin\frac{\theta}{2} e^{i\phi}.
\end{align}

\printbib


\stopmcols

\end{document}