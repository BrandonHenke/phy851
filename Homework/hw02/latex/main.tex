\documentclass[
a4paper,
10pt,
twoside,
% prd,
% aps,
% nofootinbib,
% superscriptaddress,
% floatfix,
% preprintnumbers,
]{article}

% MUST BE RUN WITH LUALATEX

\usepackage{preamble}
\usepackage{titleinfo}

\geometry{ % Set document margins
	top			= 1cm,
	bottom	= 2cm,
	left		= 1cm,
	right		= 1cm
}

\newcommand{\mcols}{2} % Choose number of columns (>= 1)


\bibSetup{refs.bib} % Give references file 
% ===== Format headers & footers =====

\pagestyle{fancy}
\fancyhf{}
\fancyhead[LE,RO]{B. Henke}
\fancyhead[LO]{\headertitle\hspace{0.5cm}\textit{QCNP}}
\fancyhead[RE]{\textit{QCNP}\hspace{0.5cm}\headertitle}
\fancyfoot[RE,LO]{\thepage}

\begin{document}
% \tableofcontents
\titleinf
\maketitle
\startmcols

\section{Pauli Matrices}
\subsection{}% A
The proof is tedious but trivial.
\subsection{}% B
The eigenvalues for all of the following are $\pm 1$.
\begin{align}
	\ket{S_x,\pm} &= \frac{1}{\sqrt{2}} \left(\ket{-}\pm\ket{+}\right).\\
	\ket{S_y,\pm} &= \frac{1}{\sqrt{2}} \left(\ket{-}\pm i\ket{+}\right).\\
	\ket{S_z,\pm} &= \ket{\pm}.
\end{align}
\subsection{}% C
The proof is tedious but trivial.
\section{Sakurai 1.4}
\subsection{}% A
Since all Pauli matrices are traceless, $\tr{X}= 2a_0$.
Additionally, $\tr{\sigma_k X} = 0 + \sum_{i=1}^3\sigma_k\sigma_i = 2a_k.$

\subsection{}% B
\begin{align}
	a_0 &= \frac{1}{2}\left(X_{00} + X_{11}\right),\\
	a_1 &= \frac{1}{2}\left(X_{10} + X_{01}\right),\\
	a_2 &= -\frac{i}{2}\left(X_{10} - X_{01}\right),\\
	a_3 &= \frac{1}{2}\left(X_{00} - X_{11}\right).
\end{align}
\section{Sakurai 1.10}
Since $\sigma_i\sigma_j = 1$ for $i=j$ and $\sigma_i\sigma_j = -\sigma_j\sigma_i$ for $i\neq j$:
\begin{align}
	\anticommutator{\sigma_i}{\sigma_j} &= 2\delta_{ij}.\\
	\commutator{\sigma_i}{\sigma_j} &= 0.\quad(i=j)\\
\end{align}
Since $-i=\sigma_3\sigma_2\sigma_1$,
\begin{align}
	\commutator{\sigma_i}{\sigma_j} &= 2\sigma_i\sigma_j,\\
	&= 2i \left[\sigma_i\sigma_j\sigma_k \sigma_3\sigma_2\sigma_1\right] \sigma_k,\\
	&= 2i \epsilon_{ijk} \sigma_k.\quad(i\neq j \neq k) 
\end{align}
Since $S_k = \frac{\hbar}{2}\sigma_k$, $\commutator{S_i}{S_j} = \frac{\hbar^2}{4}\commutator{\sigma_i}{\sigma_j}$, and the same for the anticommutator:
\begin{align}
	\therefore \commutator{S_i}{S_j} &= \hbar i \epsilon_{ijk} S_k.\\
	\anticommutator{S_i}{S_j} &= \frac{\hbar^2}{2} \delta_{ij}.
\end{align}

\section{Sakurai 1.11}
Since $n = n^k \sigma_k$ is a unit vector,
\begin{equation}
	n = \sin\theta(\cos\phi \sigma_1 + \sin\phi \sigma_2) + \cos\theta \sigma_3.
\end{equation}
Additionally
\begin{equation}
	\ev{n} = n^k \ev{\sigma_k}{\psi}.
\end{equation}

\printbib


\stopmcols

\end{document}