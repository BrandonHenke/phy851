\documentclass[
a4paper,
10pt,
twoside,
% prd,
% aps,
% nofootinbib,
% superscriptaddress,
% floatfix,
% preprintnumbers,
]{article}

% MUST BE RUN WITH LUALATEX

\usepackage{preamble}
\usepackage{titleinfo}

\geometry{ % Set document margins
	top			= 2cm,
	bottom	= 2cm,
	left		= 1cm,
	right		= 1cm
}

\newcommand{\mcols}{2} % Choose number of columns (>= 1)


\bibSetup{refs.bib} % Give references file 
% ===== Format headers & footers =====

\pagestyle{fancy}
\fancyhf{}
\fancyhead[LE,RO]{B. Henke}
\fancyhead[LO]{\headertitle\hspace{0.5cm}}
\fancyhead[RE]{\hspace{0.5cm}\headertitle}
\fancyfoot[RE,LO]{\thepage}

\begin{document}
% \tableofcontents
\titleinf
\maketitle
\startmcols

\section{Sakurai 1.5}
It's a rotation, of course it's invariant.
\begin{align}
	a_1' &= a_1\cos\phi + a_2\sin\phi.\\
	a_2' &= a_2\cos\phi - a_1\sin\phi.\\
	a_3' &= a_3.
\end{align}
This is a rotation of the vector $a = a^k\sigma_k$ through an angle of $\phi$ in the $xy$ plane ($\sigma_1\sigma_2$ plane).
\section{Sakurai 1.14}
\section{Sakurai 1.15}
\section{Sakurai 1.21}
\section{Sakurai 1.25}

\printbib


\stopmcols

\end{document}