\documentclass[
a4paper,
10pt,
twoside,
% prd,
% aps,
% nofootinbib,
% superscriptaddress,
% floatfix,
% preprintnumbers,
]{article}

% MUST BE RUN WITH LUALATEX

\usepackage{preamble}
\usepackage{titleinfo}

\geometry{ % Set document margins
	top			= 2cm,
	bottom	= 2cm,
	left		= 1cm,
	right		= 1cm
}

\newcommand{\mcols}{2} % Choose number of columns (>= 1)


\bibSetup{refs.bib} % Give references file 
% ===== Format headers & footers =====

\pagestyle{fancy}
\fancyhf{}
\fancyhead[LE,RO]{B. Henke}
\fancyhead[LO]{\headertitle\hspace{0.5cm}}
\fancyhead[RE]{\hspace{0.5cm}\headertitle}
\fancyfoot[RE,LO]{\thepage}

\begin{document}
% \tableofcontents
\titleinf
\maketitle
\startmcols

\section{Sakurai 1.5}

It's a rotation, of course it's invariant:
\begin{align}
	\det(ABC) &= \det(A)\det(B)\det(C),\\
	\therefore &\det(e^{i n \phi/2}a^k\sigma_k e^{-i n \phi/2})\\
	&= \det(e^{i n^k\sigma_k \phi/2})\det(a^k\sigma_k)\det(e^{-i n^k\sigma_k \phi/2}),\\
	&= e^{\tr(i n^k\sigma_k \phi/2)} \cdot \det(a^k \sigma_k) \cdot e^{\tr(-i n^k\sigma_k \phi/2)},\\
	&= \det(a^k \sigma_k).
\end{align}
\begin{align}
	a' &= a^k e^{i \sigma_3 \phi/2} \sigma_k e^{-i \sigma_3 \phi/2},\\
	&= a^k \left(\cos\frac{\phi}{2} + i\sigma_3 \sin\frac{\phi}{2}\right) \left(\sigma_k\cos\frac{\phi}{2} - i\sigma_k\sigma_3 \sin\frac{\phi}{2}\right),\\
	&= a^k \left(\sigma_k\cos^2\frac{\phi}{2} + i[\sigma_3,\sigma_k] \sin\frac{\phi}{2}\cos\frac{\phi}{2}\right.\\
	&\qquad+ \left. \sigma_3\sigma_k\sigma_3 \sin^2 \frac{\phi}{2} \right).
\end{align}
If $k = 3$, $\commutator{\sigma_3}{\sigma_k} = 0$, and $\sigma_3^3 = \sigma_3$.
Thus,
\begin{align}
	a' &= a^1 \left(\sigma_1\cos^2\frac{\phi}{2} + i[\sigma_3,\sigma_1] \sin\frac{\phi}{2}\cos\frac{\phi}{2}\right.\nonumber\\
	&\qquad+ \left. \sigma_3\sigma_1\sigma_3 \sin^2 \frac{\phi}{2} \right)\nonumber\\
	&\quad+ a^2 \left(\sigma_2\cos^2\frac{\phi}{2} + i[\sigma_3,\sigma_2] \sin\frac{\phi}{2}\cos\frac{\phi}{2}\right.\nonumber\\
	&\qquad+ \left. \sigma_3\sigma_2\sigma_3 \sin^2 \frac{\phi}{2} \right)\nonumber\\
	&\quad+ a^3\sigma_3,\\
	%
	&= a^1 \left(\sigma_1\cos^2\frac{\phi}{2} - \sigma_2 \sin\phi - \sigma_1 \sin^2 \frac{\phi}{2} \right)\nonumber\\
	&\quad+ a^2 \left(\sigma_2\cos^2\frac{\phi}{2} + \sigma_1 \sin\phi - \sigma_2 \sin^2 \frac{\phi}{2} \right)\nonumber\\
	&\quad+ a^3\sigma_3,\\
	&= a^1 \left(\sigma_1\cos\phi - \sigma_2 \sin\phi\right)\nonumber\\
	&\quad+ a^2 \left(\sigma_2\cos\phi + \sigma_1 \sin\phi\right)\nonumber\\
	&\quad+ a^3\sigma_3,\\
	&= (a^1\cos\phi + a^2 \sin\phi)\sigma_1 \nonumber\\
	&\quad+ (-a^1\sin\phi + a^2\cos\phi) \sigma_2 \nonumber\\
	&\quad+ a^3\sigma_3.
\end{align}
\begin{align}
	a'^1 &= a^1\cos\phi + a^2 \sin\phi.\\
	a'^2 &= -a^1\sin\phi + a^2\cos\phi.\\
	a'^3 &= a^3.
\end{align}
This is a rotation of the vector $a = a^k\sigma_k$ through an angle of $\phi$ in the $xy$ plane ($\sigma_1\sigma_2$ plane).
\section{Sakurai 1.14}
\subsection{} %A
\begin{align}
	\abs{\ip{S_x;+}{\psi}}^2 &= \abs{\frac{1}{\sqrt{2}}\left(\bra{+}+\bra{-}\right) \left( \cos\frac{\gamma}{2}\ket{+} \pm \sin\frac{\gamma}{2} \ket{-} \right)}^2,\\
	&= \frac{1}{2}\left(\cos\frac{\gamma}{2} \pm \sin\frac{\gamma}{2}\right)^2,\\
	&= \frac{1}{2}\left(1 \pm \sin\gamma \right).
\end{align}

\subsection{} %B
\begin{align}
	\ev{(S_x-\ev{S_x})^2} &= \ev{S_x^2} - \ev{S_x}^2.\\
	\ev{S_x^2} &= \frac{\hbar^2}{4} \left(\cos^2\frac{\gamma}{2} + \sin^2 \frac{\gamma}{2}\right),\\
	&= \frac{\hbar^2}{4}.\\
	\ev{S_x}^2 &= \frac{\hbar^2}{4} 4\sin^2\frac{\gamma}{2}\cos^2\frac{\gamma}{2},\\
	&= \frac{\hbar^2}{4} \sin^2 \gamma.\\
	\ev{S_x^2} - \ev{S_x}^2 &= \frac{\hbar^2}{4} \cos^2 \gamma.
\end{align}

\section{Sakurai 1.15}
This is the same as polarization of photons, but with twice the angle:
the intensity of the final $s_z = -\hbar/2$ beam is $I(\beta) = \sin[2](\beta)/4$.

The intensity of the final $s_z = -\hbar/2$ beam is maximized when the second device is measuring anywhere in the $xy$-plan: $\beta = \pi/2$.

\section{Sakurai 1.21}
\subsection{} %A
\begin{align}
	\ev{S_x^2} - \ev{S_x}^2 &= \frac{\hbar^2}{4} - 0,\\
	&= \frac{\hbar^2}{4}.
	\ev{S_y^2} - \ev{S_y}^2 &= \frac{\hbar^2}{4}.\\
	\frac{1}{4} \abs{\ev{\commutator{S_x}{S_y}}}^2 &= \frac{\hbar^4}{16}.\\
	\ev{(\Delta S_x)^2}\ev{(\Delta S_y)^2} &\geq \frac{1}{4} \abs{\ev{\commutator{S_x}{S_y}}}^2,\\
	\frac{\hbar^4}{16} &\geq \frac{\hbar^4}{16}. \qquad \cmark
\end{align}
\subsection{} %B
\begin{align}
	\ev{S_x^2} - \ev{S_x}^2 &= \frac{\hbar^2}{4} - \frac{\hbar^2}{4},\\
	&= 0.\\
	\frac{1}{4} \abs{\ev{\commutator{S_x}{S_y}}}^2 &= 0.\\
	\ev{(\Delta S_x)^2}\ev{(\Delta S_y)^2} &\geq \frac{1}{4} \abs{\ev{\commutator{S_x}{S_y}}}^2,\\
	0 &\geq 0. \qquad \cmark
\end{align}

\section{Sakurai 1.25}
\subsection{} %A
The operator, $B$, exhibits a degenerate spectrum (repeated eigenvalues):
\begin{align}
	\det(\lambda-B) = 0 &= (\lambda-b)(\lambda^2-b^2),\\
	\lambda &\in \left\{ b,b,-b \right\}.
\end{align}

\subsection{} %B
\begin{align}
	AB &= \begin{pmatrix}
		a & 0 & 0 \\
		0 & -a & 0 \\
		0 & 0 & -a 
	\end{pmatrix}\begin{pmatrix}
		b & 0 & 0 \\
		0 & 0 & -ib \\
		0 & ib & 0 
	\end{pmatrix},\\
	&= \begin{pmatrix}
		ab & 0 & 0 \\
		0 & 0 & iab \\
		0 & -iab & 0 
	\end{pmatrix},\\
	&= \begin{pmatrix}
		b & 0 & 0 \\
		0 & 0 & -ib \\
		0 & ib & 0 
	\end{pmatrix}\begin{pmatrix}
		a & 0 & 0 \\
		0 & -a & 0 \\
		0 & 0 & -a 
	\end{pmatrix},\\
	&= BA.
\end{align}
\subsection{} %C
The eigenvalues of $AB$ are $\lambda_{ab} \in \left\{ ab,ab, -ab\right\}$.
\begin{align}
	\begin{pmatrix}
		ab & 0 & 0 \\
		0 & 0 & iab \\
		0 & -iab & 0 
	\end{pmatrix}\begin{pmatrix}
		v^1 \\ v^2 \\ v^3
	\end{pmatrix} &= \lambda_{ab}\begin{pmatrix}
		v^1 \\ v^2 \\ v^3
	\end{pmatrix},\\
	ab v^1 &= \lambda_{ab} v^1,\\
	iab v^3 &= \lambda_{ab} v^2,\\
	-iab v^2 &= \lambda_{ab} v^3.\\
	\rightarrow v&\in \left\{
		\begin{pmatrix}
			1 \\ 0 \\ 0
		\end{pmatrix},
		\frac{1}{\sqrt{2}}\begin{pmatrix}
			0 \\ i \\ 1
		\end{pmatrix},
		\frac{1}{\sqrt{2}}\begin{pmatrix}
			0 \\ -i \\ 1
		\end{pmatrix}
	\right\}.
\end{align}
The respective eigenvalues are given above.

\printbib


\stopmcols

\end{document}