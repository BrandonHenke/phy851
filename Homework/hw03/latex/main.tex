\documentclass[a4paper,12pt,twoside]{article}

\usepackage{preamble}

\title{Homework 02}
\author{Brandon Henke\\PHY851\\Scott Pratt}
\date{September 20, 2021}

\setcounter{section}{1}

\fancyhead[LE,RO]{B. Henke}
\fancyhead[RE,LO]{\thepage}

\begin{document}
%\tableofcontents

\maketitle
\begin{multicols*}{2}

\setcounter{subsection}{9}
\subsection{}%10
\subsubsection{}
\begin{align}
	\ket{45} &= \frac{1}{\sqrt{2}} \left( \ket{R} + i\ket{L} \right),\\
	\ket{135} &= \frac{1}{\sqrt{2}} \left( -\ket{R} + i\ket{L} \right).
\end{align}
\subsubsection{}
\begin{align}
	\frac{e^{-i\pi/4}}{2} \begin{pmatrix}
		\frac{1}{\sqrt{2}} & -\frac{i}{\sqrt{2}} \\
		\frac{1}{\sqrt{2}} & \frac{i}{\sqrt{2}} \\
	\end{pmatrix}&
	\begin{pmatrix}
		\frac{1}{\sqrt{2}} & -\frac{1}{\sqrt{2}} \\
		\frac{1}{\sqrt{2}} & \frac{1}{\sqrt{2}} \\
	\end{pmatrix}\nonumber\\
	&=
	\begin{pmatrix}
		-\frac{i}{\sqrt{2}} & -\frac{1}{\sqrt{2}} \\
		\frac{1}{\sqrt{2}} & \frac{i}{\sqrt{2}}
	\end{pmatrix}
\end{align}
The factor of $e^{-i\pi/4}$ is there to rotate the state into a nicer (but identical) form, since relative phase is all that matters.
\subsubsection{}
For a unitary transformation, the adjoint is equal to the inverse.
\begin{align}
	\begin{pmatrix}
		-\frac{i}{\sqrt{2}} & -\frac{1}{\sqrt{2}} \\
		\frac{1}{\sqrt{2}} & \frac{i}{\sqrt{2}}
	\end{pmatrix}&
	\begin{pmatrix}
		\frac{i}{\sqrt{2}} & \frac{1}{\sqrt{2}} \\
		-\frac{1}{\sqrt{2}} & -\frac{i}{\sqrt{2}}
	\end{pmatrix}\nonumber\\
	&= \begin{pmatrix}
		1 & 0 \\
		0 & 1
	\end{pmatrix}.
\end{align}
Since it has been shown that the adjoint is the transformation's inverse, the transformation is unitary.
\subsection{}%11
\subsubsection{}
\begin{equation}
	P_x = \op{X} = \begin{pmatrix}
		1 & 0 \\
		0 & 0
	\end{pmatrix}.
\end{equation}
\subsubsection{}
The eigenvalues of $P_x$ are
\begin{equation}
	\lambda = 1,0.
\end{equation}
The eigenstates are
\begin{equation}
	\ket{\lambda} = \ket{X},\ket{Y},
\end{equation}
respectively.
\subsubsection{}
In the $RL$ basis, $P_x$ is given by
\begin{equation}
	P_x = \frac{1}{2} \begin{pmatrix}
		1 & 1 \\ 1 & 1
	\end{pmatrix} = \frac{1}{2}(\mathbb{1}+\sigma_1).
\end{equation}
From this, it's easy to see that the eigenvalues are, again,
\begin{equation}
	\lambda = 1,0.
\end{equation}

\subsection{}%12
\subsubsection{}
\begin{align}
	\Tr(U^\dagger A U) &= U^*_{ij}A_{jk}U_{ki},\\
	&= U^*_{ij}U_{ki}A_{jk},\\
	&= \delta_{jk} A_{jk},\\
	&= A_{jj} = \Tr(A).
\end{align}
\subsubsection{}
\begin{align}
	\Tr(AB) &= A_{ij}B_{ji},\\
	&= B_{ji}A_{ij},\\
	&= \Tr(BA).
\end{align}
\subsection{}%13
Let $\phi(z,t)$ be the operator
\begin{equation}
	\phi(z,t) = e^{-i\omega t} \begin{pmatrix}
		e^{ik_x z} & 0 \\
		0 & e^{-ik_y z}
	\end{pmatrix},
\end{equation}
where $z$ is the distance into the crystal.
Then
\begin{align}
	\phi(z,t)\ket{\psi} &= \frac{1}{\sqrt{2}} e^{-i\omega t} \begin{pmatrix}
		e^{ik_x z}\\ e^{ik_y z}
	\end{pmatrix},\\
	&= \frac{1}{\sqrt{2}} \begin{pmatrix}
		1\\ e^{i(k_y-k_x) z}
	\end{pmatrix}
\end{align}
is the polarization of the photon at a distance $z$ inside the crystal.
If the photon comes out as righthand circularly polarized, then
\begin{equation}
	e^{i(k_y-k_x)z} = i,
\end{equation}
or
\begin{equation}
	(k_y-k_x)z = \frac{\pi}{2}.
\end{equation}
Thus
\begin{equation}
	z = \frac{c}{4\abs{n_y-n_x} \nu} =
\end{equation}
\subsection{}%14
\subsection{}%15
\subsubsection{}
\subsubsection{}
\subsubsection{}
\subsubsection{}

\setcounter{subsection}{16}
\subsection{}%17
\subsubsection{}
\subsubsection{}
\subsubsection{}
\subsubsection{}
\subsection{}%18

\setcounter{section}{2}
\setcounter{subsection}{0}
\subsection{}%1
\subsection{}%2
\subsection{}%3

\citation
\bibdata
\bibstyle



% \nocite{*}
% \printbibliography[title={References},heading=bibnumbered]


\end{multicols*}
\end{document}
