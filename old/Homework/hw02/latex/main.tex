\documentclass[a4paper,12pt,twoside]{article}

\usepackage{preamble}

\title{Homework 02}
\author{Brandon Henke\\PHY851\\Scott Pratt}
\date{September 20, 2021}

\setcounter{section}{1}

\fancyhead[LE,RO]{B. Henke}
\fancyhead[RE,LO]{\thepage}

\begin{document}
%\tableofcontents

\maketitle
% \begin{multicols*}{2}

\setcounter{subsection}{9}
\subsection{}%10
\subsubsection{}
\begin{align}
	\ket{45} &= \frac{1}{\sqrt{2}} \left( \ket{R} + i\ket{L} \right),\\
	\ket{135} &= \frac{1}{\sqrt{2}} \left( -\ket{R} + i\ket{L} \right).
\end{align}
\subsubsection{}
\begin{align}
	\frac{e^{-i\pi/4}}{2} \begin{pmatrix}
		\frac{1}{\sqrt{2}} & -\frac{i}{\sqrt{2}} \\
		\frac{1}{\sqrt{2}} & \frac{i}{\sqrt{2}} \\
	\end{pmatrix}&
	\begin{pmatrix}
		\frac{1}{\sqrt{2}} & -\frac{1}{\sqrt{2}} \\
		\frac{1}{\sqrt{2}} & \frac{1}{\sqrt{2}} \\
	\end{pmatrix}\nonumber\\
	&=
	\begin{pmatrix}
		-\frac{i}{\sqrt{2}} & -\frac{1}{\sqrt{2}} \\
		\frac{1}{\sqrt{2}} & \frac{i}{\sqrt{2}}
	\end{pmatrix}
\end{align}
The factor of $e^{-i\pi/4}$ is there to rotate the state into a nicer (but identical) form, since relative phase is all that matters.
\subsubsection{}
For a unitary transformation, the adjoint is equal to the inverse.
\begin{align}
	\begin{pmatrix}
		-\frac{i}{\sqrt{2}} & -\frac{1}{\sqrt{2}} \\
		\frac{1}{\sqrt{2}} & \frac{i}{\sqrt{2}}
	\end{pmatrix}&
	\begin{pmatrix}
		\frac{i}{\sqrt{2}} & \frac{1}{\sqrt{2}} \\
		-\frac{1}{\sqrt{2}} & -\frac{i}{\sqrt{2}}
	\end{pmatrix}\nonumber\\
	&= \begin{pmatrix}
		1 & 0 \\
		0 & 1
	\end{pmatrix}.
\end{align}
Since it has been shown that the adjoint is the transformation's inverse, the transformation is unitary.
\subsection{}%11
\subsubsection{}
\begin{equation}
	P_x = \op{X} = \begin{pmatrix}
		1 & 0 \\
		0 & 0
	\end{pmatrix}.
\end{equation}
\subsubsection{}
The eigenvalues of $P_x$ are
\begin{equation}
	\lambda = 1,0.
\end{equation}
The eigenstates are
\begin{equation}
	\ket{\lambda} = \ket{X},\ket{Y},
\end{equation}
respectively.
\subsubsection{}
In the $RL$ basis, $P_x$ is given by
\begin{equation}
	P_x = \frac{1}{2} \begin{pmatrix}
		1 & 1 \\ 1 & 1
	\end{pmatrix} = \frac{1}{2}(\mathbb{1}+\sigma_1).
\end{equation}
From this, it's easy to see that the eigenvalues are, again,
\begin{equation}
	\lambda = 1,0.
\end{equation}

\subsection{}%12
\subsubsection{}
\begin{align}
	\Tr(U^\dagger A U) &= U^*_{ij}A_{jk}U_{ki},\\
	&= U^*_{ij}U_{ki}A_{jk},\\
	&= \delta_{jk} A_{jk},\\
	&= A_{jj} = \Tr(A).
\end{align}
\subsubsection{}
\begin{align}
	\Tr(AB) &= A_{ij}B_{ji},\\
	&= B_{ji}A_{ij},\\
	&= \Tr(BA).
\end{align}
\subsection{}%13
Let $\phi(z,t)$ be the operator
\begin{equation}
	\phi(z,t) = e^{-i\omega t} \begin{pmatrix}
		e^{ik_x z} & 0 \\
		0 & e^{-ik_y z}
	\end{pmatrix},
\end{equation}
where $z$ is the distance into the crystal.
Then
\begin{align}
	\phi(z,t)\ket{\psi} &= \frac{1}{\sqrt{2}} e^{-i\omega t} \begin{pmatrix}
		e^{ik_x z}\\ e^{ik_y z}
	\end{pmatrix},\\
	&= \frac{1}{\sqrt{2}} \begin{pmatrix}
		1\\ e^{i(k_y-k_x) z}
	\end{pmatrix}
\end{align}
is the polarization of the photon at a distance $z$ inside the crystal.
If the photon comes out as righthand circularly polarized, then
\begin{equation}
	e^{i(k_y-k_x)z} = i,
\end{equation}
or
\begin{equation}
	(k_y-k_x)z = \frac{\pi}{2}.
\end{equation}
Thus
\begin{equation}
	z = \frac{c}{4\abs{n_y-n_x} \nu}.
\end{equation}
\subsection{}%14
Use geometric algebra!
\begin{align}
	R(\phi)\sigma_1 R^{-1}(\phi)
	&= e^{-i\sigma_{3}\phi/2}\sigma_1 e^{i\sigma_{3}\phi/2},\\
	&= \left( \cos(\phi/2)-i\sin(\phi/2)\sigma_3 \right)\sigma_1\left( \cos(\phi/2)+i\sin(\phi/2)\sigma_3 \right),\\
	&= \left( \cos(\phi/2)\sigma_1+\sin(\phi/2)\sigma_2\sigma_1\sigma_1 \right)\left( \cos(\phi/2)+\sin(\phi/2)\sigma_1\sigma_2 \right),\\
	&= \cos[2](\phi/2)\sigma_1+2\cos(\phi/2)\sin(\phi/2)\sigma_2 - \sin[2](\phi/2)\sigma_1,\\
	&= \sigma_1 \cos(\phi) + \sigma_2 \sin(\phi).
\end{align}
\subsection{}%15
\subsubsection{}
\begin{align}
	\ket{x,+} &= \frac{1}{\sqrt{2}}\left(\ket{\uparrow}+\ket{\downarrow}\right),\\
	\ket{x,-} &= \frac{1}{\sqrt{2}}\left(\ket{\uparrow}-\ket{\downarrow}\right),\\
	\ket{y,+} &= \frac{1}{\sqrt{2}}\left(\ket{\uparrow}+i\ket{\downarrow}\right),\\
	\ket{y,-} &= \frac{1}{\sqrt{2}}\left(\ket{\uparrow}-i\ket{\downarrow}\right).
\end{align}
\subsubsection{}
\begin{align}
	\op{z,+} &= \begin{pmatrix}
		1 & 0 \\ 0 & 0
	\end{pmatrix},\\
	\op{z,-} &= \begin{pmatrix}
		0 & 0 \\ 0 & 1
	\end{pmatrix},\\
	\op{x,+} &= \frac{1}{2}\begin{pmatrix}
		1 & 1 \\ 1 & 1
	\end{pmatrix},\\
	\op{x,-} &= \frac{1}{2}\begin{pmatrix}
		1 & -1 \\ -1 & 1
	\end{pmatrix},\\
	\op{y,+} &= \frac{1}{2}\begin{pmatrix}
		1 & -i \\ i & 1
	\end{pmatrix},\\
	\op{y,-} &= \frac{1}{2}\begin{pmatrix}
		1 & i \\ -i & 1
	\end{pmatrix}.
\end{align}
\subsubsection{}
\begin{equation}
	\rho_{60/40} = 0.6\op{z,+}+0.4\op{z,-} = \begin{pmatrix}
		0.6 & 0 \\ 0 & 0.4
	\end{pmatrix}
\end{equation}
\subsubsection{}
\begin{align}
	\ev{S_z}{y,+} &= \frac{1}{2}\Tr(\op{y,+}S_z),\\
	&= 0.
\end{align}

\setcounter{subsection}{16}
\subsection{}%17
\subsubsection{}
\begin{align}
	e^{-i\hat{H}t/\hbar} &= e^{-i\left(\hat{1}\frac{m_\mu+m_\tau}{2}+\alpha\sigma_1+\frac{m_\mu-m_\tau}{2}\sigma_3\right)t/\hbar},\\
	&= e^{-i(m_\mu+m_\tau)t/(2 \hbar)}e^{-i\left(\alpha\sigma_1+\frac{m_\mu-m_\tau}{2}\sigma_3\right)t/\hbar},\\
	&= e^{-i(m_\mu+m_\tau)t/(2 \hbar)}e^{-i\sigma_n\omega t},\\
	&= e^{-i(m_\mu+m_\tau)t/(2 \hbar)}\left(\cos(\omega t)-i\sigma_n\sin(\omega t)\right).
\end{align}
\subsubsection{}
\begin{align}
	P_{\mu\rightarrow\tau} &= \abs{\bra{\tau}e^{-i\hat{H}t/\hbar}\ket{\mu}}^2,\\
	&= \sin[2](\omega t)\abs{\bra{\tau}\sigma_n\ket{\mu}}^2,\\
	&= \frac{\alpha^2}{\hbar^2\omega^2}\sin[2](\omega t).
\end{align}
\subsubsection{}
\begin{equation}
	\tau_0 = \frac{\pi}{\omega}.
\end{equation}
\subsubsection{}
\begin{equation}
	\tau = \gamma\tau_0 \approx \frac{\hbar k c}{m}\tau_0
\end{equation}
\subsection{}%18
\begin{align}
	\Tr(A_H(t)B_H(t)C_H(t))
	&= \Tr(U(t)A_SU^\dagger(T)U(t)B_SU^\dagger(T)U(t)C_SU^\dagger(T)),\\
	&= \Tr(U(t)A_SB_SC_SU^\dagger(T)),\\
	&= \Tr(A_SB_SC_S).
\end{align}

\setcounter{section}{2}
\setcounter{subsection}{0}
\subsection{}%1
Operating on the momentum eigenstates with the position operator doesn't make sense.
If the momentum is clearly defined ($\hat{p}\ket{q} = q\ket{q}$), then the position isn't defined.
\subsection{}%2
It's sufficient to show $\ev{\hat{p}^2}\geq0$:
\begin{align}
	\ev{\hat{p}^2}
	&= \hbar^2 \int \dd{x} \partial_x\psi^*(x)\partial_x\psi(x),\\
	&= \hbar^2 \int \dd{x} \abs{\partial_x\psi(x)}^2 \geq 0.
\end{align}
\subsection{}%3
The solutions for the problem when \textbf{shifted} by $a$ to the right, in each region, are
\begin{align}
	\psi_I(x) &= e^{kx},\\
	\psi_{II}(x) &= A\cos(k_{II}x) + B\sin(k_{II}x),\\
	\psi_{III}(x) &= Ce^{-kx},
\end{align}
where
\begin{equation}
	k^2 = \frac{2m\abs{E}}{\hbar^2}
\end{equation}
and
\begin{equation}
	k_{II}^2 = \frac{2m(V_0-\abs{E})}{\hbar^2}.
\end{equation}
From the boundary conditions,
\begin{align}
	A &= 1,\\
	B &= \frac{k}{k_{II}},\\
	C &= e^{2ka}\left(\cos(2k_{II}a)+B\sin(2k_{II}a)\right).
\end{align}
These are valid for $\abs{E} \geq 0$, and
\begin{align}
	0 &= \tan(2k_{II}a) ,\\
	n\pi &= 2k_{II}a,\\
	\abs{E} &= V_0 - \frac{n^2\pi^2\hbar^2}{8a^2m},\\
	\therefore V_0 &\geq \frac{n^2\pi^2\hbar^2}{8a^2m}.
\end{align}

\citation
\bibdata
\bibstyle



% \nocite{*}
% \printbibliography[title={References},heading=bibnumbered]


% \end{multicols*}
\end{document}
